\section{Exercise 2}
\lstinputlisting{Ex2.py}
\subsection{Exercise 2a}
In order to build an octree, we will make a class that stores each node, and at the lowest level the leaves. In this node, we have to store 8 child nodes (except for the leaves), the depth, box center, total mass, and the center of mass. To build the tree, we will use a recursive algorithm where we first check if the current depth of node is larger than the maximum depth given as input. Next, we calculate which particles are in which new octant. We do this by taking `positions\_particles $>=$ box\_center'. Based on the combinations of True/False for the x-y-x-coordinates, we give the particle an index with the following line: index\_octant = $(\mathrm{mask[:,0] << 2)\ |\ (mask[:,1]) << 1\ |\ (mask[:,2] << 0)}$, where the mask is based on the position with relevance to the center. After this, we will change the index array with all the particles indices. For example, particle indices that are in the octant with index 0 are stored in the first part of the index array, and particles in octant with index 1 are stored in the second part of the index array. In order to make this work, we will also save start index for each new octant with the length (number of particles in the specific octant. Now we calculate the new box centers and store them too in the node class. 

To create the mass maps at different slices $x_i$, we will traverse the tree and create a 3D mass map. It will go down the tree until the required depth is reached. At this depth, it calculates the index based on its box center. When we have the whole map, we can easily select different slices of the tree. The results are shown in Figure \ref{fig:2a1}, \ref{fig:2a2}, and \ref{fig:2a3}.

\begin{figure}[h!]
  \centering
  \includegraphics[width=0.9\linewidth]{fig2a_level3.png}
  \caption{The mass maps for the four first slices in x at a depth of 3.}
  \label{fig:2a1}
\end{figure}

\begin{figure}[h!]
  \centering
  \includegraphics[width=0.9\linewidth]{fig2a_level5.png}
  \caption{The mass maps for the four first slices in x at a depth of 5.}
  \label{fig:2a2}
\end{figure}

\begin{figure}[h!]
  \centering
  \includegraphics[width=0.9\linewidth]{fig2a_level7.png}
  \caption{The mass maps for the four first slices in x at a depth of 7.}
  \label{fig:2a3}
\end{figure}

\subsection{Exercise 2b}
\lstinputlisting{Fourier.py}
For exercise 2b, we will perform a Fourier transform to find the potential. We first create a density map. To do this, we will use the mass map at depth 7 found in the previous exercise. To find the density, we simply divide by $(L/2^7)^3$, as this is the size of the leaves at depth 7. Next, we will perform a recursive FFT along all three axes of the cube. In order to get the 3D FFT, we can apply the FFT along the first axis, after which we do the second and third axis. Now we define $k^2$, because we have to divide by this before we do the inverse Fourier transform. To find k in one dimension, we take an array ranging from $-N_{grid}\ //\ 2 $ to $N_{grid}\ // \ 2$, where Ngrid is 128 in this case. Now we multiply this array with $2\pi$ and divide by $L \cdot d \cdot N_{grid}$, where $d = \frac{N_{grid}}{L}$. Therefore, this is equal to multiplying with $2\pi$ and dividing by $N_{grid}^2$. Now we can calculate $k^2$ by making a meshgrid of k and calculating $k^2 = k_x^2 + k_y^2 + k_z^2$. The next step is to divide the Fourier transformed density map by $k^2$ and taking the inverse FFT. We multiply this by $-G/2\pi$ to finally find the potential. The potential at different slices $x_i$ is shown in Figure \ref{fig:2b} 

\begin{figure}[h!]
  \centering
  \includegraphics[width=0.9\linewidth]{fig2b.png}
  \caption{The Gravitational potential at 4 different x slices calculated with the FFT.}
  \label{fig:2b}
\end{figure}
